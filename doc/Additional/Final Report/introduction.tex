u\documentclass{article}
\begin{document}

\section{Introduction}

Blocks World For Teams, of BW4T for short, is an environment for multi-agent systems. This is a system for joint activity. Multiple agents are programmed to work together as a team with the same goal. They have to divide the workload in order to even be able to solve the problem, or solve the problem more efficiently. The BW4T environment offers a, for humans, quite simple problem. There is a number of rooms that contain coloured blocks and there is a sequence that tells the order in which the blocks should be collected. The simplicity of BW4T is its strength. It is powerful in what is does. This is a good point to start testing teams that have to work together on an operation. However, it is nowhere near any real world problems that require real human teams. 

\subsection{Problem Description}
The real world immediately brings us to the problem we are trying to address. When real rescue operations are done, they are executed by skilled and very well trained teams. They have to be prepared for any kind of situation, with as their main goal to rescue the people that are in danger. Even if the rescue teams are very well trained unsolvable situations might still occur. Teams described so far only consisted of either agents or humans. A combination of the two could be the solution to these problems. Robots can be made in all kinds of sizes equipped with different capabilities, just needed for specific tasks. This way they might be able to reach places humans cannot access. Thus human-robot interaction is very important. In BW4T this is not possible advanced enough. Also there is only one kind of robot available in the current version. A more realistic approach to the problem is needed.
\subsection{End-user's Requirements}
In order to extend the possibilities in the Blocks World For Teams the customer gave a set of requirements of what they would like to see in the next version. These requirements serve as a goal to what needs to be done in the course of this project. There are a couple of main requirements that needed to be addressed. The first thing was that the coupling of the software was to high. This sharply reduces maintainability of the software. So is becomes more difficult to add or remove features. A great restructure is needed to reduce the coupling and increase cohesion. The second requirement was for the environment to become a step closer to realistic scenarios. For this to happen a bot and environment store are needed to increase realism. Bots in the environment should be customizable. For example their sizes need to be adjustable and should have different capabilities like seeing colors or being able to actually pick up blocks. Also the maps on which the simulations run should become more advanced and editable. The current approach to map creation is too basic. The third requirement was human-robot interaction. The customer asked for a completely new interface for interacting with the robots. This is needed for rescue operations in the future hat use both humans and robots.


\end{document}
