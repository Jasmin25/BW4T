\documentclass{article}
\usepackage{graphicx}
\usepackage{geometry}

\geometry{verbose,a4paper,tmargin=2cm,bmargin=2cm,lmargin=2cm,rmargin=2cm}


\begin{document}

\section*{Outlook}

\subsection*{HumanPlayer GUI}
In the current setup of the BW4T environment blocks can be dropped outside a room, but will then disappear from the screen. The environment doesn't provide functions for showing, picking up and communicating about blocks outside rooms. Keeping the visual aspect of the blocks outside a room is a fairly easy fix, implementing the needed functionality to be able to pick blocks back up and communicate about these blocks asks for bigger changes throughout BW4T. \\
\\
The messaging system of the current BW4T environment only allows robots to communicate the most basic commands like the color of a block or the location of a robot. This also means robots are only able to carry out basic tasks. Adding more advanced messages opens up a wide variety of possibilities for more advanced tasks for bots.

\subsection*{Path Planning}
In the BW4T environment robots plan their path by navigating through the zones on the map. The navigation algorithm creates a graph consisting of all zones, and calculates the shortest path from the start zone to the destination zone. Each zone has a location on the map, however, this location specifies where the center of the zone is. As a result, when paths are calculated using the zones location, the path always goes through the center of the zone. \\
The new version of BW4T includes collision detection and obstacle avoidance. When navigating around obstacles a new path planner is used, which unlike the usual path planner does not navigate over zones, but points (coordinates) on the map. Due to the higher precision the robot can easily navigate around obstacles in its way. The downside of this path planner is that the algorithm is computationally expensive. A graph with $N$ vertices and up to $N^4$ has to be generated, with N being the product of the width and height of the map (in points). As such the path planner can not be used often as it would significantly slow down the simulation. \\
Since the paths by the original path planner all go through the center of zones, the chance is fairly high that a robot will collide with another bot. While this can be safely navigated, it does not realistically simulate a corridor where more than one bot can concurrently pass through. Due to time constraints this problem remains unsolved, but three possible solutions will be presented. \\
\\
The first solution is to, when planning a path through the zones, return a random point in the zone as opposed to the center of the zone. Because each bot will (most likely) have a different path, collisions will be avoided to an extent. The disadvantage of this method is that the paths will not be elegant. A bot may have to make like a snake to go to a location when a straight line would have been more adequate. Especially with the addition of the battery, which is used when the bot is moving, this may result in an inaccurate simulation of battery use since the bot would be using more energy than is strictly needed. \\
The second solution, albeit more complicated to implement, is to select a random point in the zone, and make sure that the points selected in all subsequent zones have (if possible) either the same X coordinate or Y coordinate. This would result in straighter lines, lowering the battery use and reducing the chance of collisions when criss-crossing between zones. \\
The third solution is to change change the way robots navigate through the map. By only planning a small part of the path at a time and taking other robots into consideration while planning it's path a robot could find the shortest possible path to it's destination. This however comes with a downside as it could consume a lot of computation power and slow down the simulation.


\end{document}
