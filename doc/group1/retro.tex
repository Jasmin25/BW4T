\documentclass{article}
\usepackage{graphicx}

\begin{document}

\title{Retro}
\maketitle

\section*{Klant review}
Heel blij met demo. Alleen: 
Koen wat niet wat het nu precies van ons moet verwachten. 
Botstore - env was mooi. 

\subsection*{Groep 1}
Afgelopen week: 
Sonar issues gefixt, client gui drop down menu gefixt. Logger verbeterd, bijna klaar. Test geschreven voor messaging in client (ook een beetje omgeschreven). Checkstyle is aangepast, zodat er maar 1 algemene file wordt gebruikt (zijn ook wat regels aangepast: 
- strings mogen vaker terug komen 
- } else structure 
Tip! Checkstyle weet niet alles, blijf zelf nadenken! 

Volgende week: 
nader te bepalen. 

\subsection*{Groep 2}
bezig geweest met botstore en handicaps, tests geschreven, integratie met groep 3 gedaan, zodat je de bots echt kan aanmaken. Ontwerp gemaakt voor de environmentstore. Wordt opgestuurd naar de klant. Nog geen tijd gehad om de gui te maken. 

Volgende week: 
Bezig met de gui van de environmentstore

Issues:
Integratie/merge viel tegen

\subsection*{Groep 3}
Scenaria editor afgemaakt, begonnen aan batch runs, vatch runs valt mee. Veel tijd kwijt aan integratie botstore/epartner gefixt. 

Volgende week: 
Human player + bat runs

issues: 


Opmerking Martin: 
Zorg dat je makkelijk kan merge. Misschien vaste tijd afspreken? Donderdag ochtend 9 uur wordt er gemerched. 

\subsection*{Valentine}
Gister avond veel doorgewerkt. Client was volledig vergeten. Dat was veel werk en het is handig voor de environment store humanbotstore de client server en core bijhouden. De structuur werd misschien niet helemaal goed begrepen. Presentatie vanuit groep 1. Gebeurt vrijdag ochtend. Na de bachelli meetings. 

\subsection{Joop}
Veel bezig geweest met communiceren. Liep tegen aan dat hij heel duidelijk moet hebben wat we moeten hebben. Zowel klant is team was daar niet heel goed in. Heeft specifieke vragen gesteld en daar ook goede informatie uitgekregen. Langs koen gegaan voor batchrun. Scenario editeror heeft mas2g exporter. Katelijn gesproken over hoe het project gaat. In principe vond ze het goed, maar hoe met het straks verder? Niet alleen documentatie maar ook verklaring van keuzes, waaom? 
Wie heeft de leerdoelen nog niet ingevuld, iedereen moet dit doen. Whatsapp onenigheid over tijden. Zorg dat je de 28 uur haalt. 
Hoe moeten we de scrum dingen aanpakken omdat er dingen best wel mis zijn gegaan: zorg dat de commitment er blijft en dan je elke keer een product blijft afleveren. 
Javadoc: author lijkt niet handig volgens martin. Levert alleen maar problemen waar leg je de grens. Xander wil het liever niet omdat we begonnen zijn met code van andere mensen. 
Presentatie opnemen. 

\end{document}