\section{Persistent Data Management}
In the Blocks World for Teams project there is no need to use databases as \gls{logfile} and map files are the only persistent data.\\

At this moment, the log files are raw dumps of the actions of a bot (might be human, might be agent). These actions vary from walking to a spot in the map to interacting with an item. \\
Currently, there is no use for these log files besides \gls{debugging}, but the log files are too poorly organised and contain insufficient information to be of significant value for this. \\
In the future, these log files might be used to reconstruct a series of actions or events. You would be able to feed a log file to the client, which will run the bot according to the lines in the log file. \\
This could become very useful if you want to analyse why a bot made certain decisions, or why it did not do what you expected it to do. \\

The map files are fairly large \gls{xmlfile}, which consist of a decleration of the map. \\
The XML file needs to contain at least the following:
\begin{itemize}
\item The size of the map (an x and y value)
\item The bot entities (including an (x,y) starting position)
\item The zones (including the name of the zone, the colour of the boxes inside, its neighbouring zones and an (x,y) value)
\item The sequence of coloured blocks to be collected (or alternatively set to be random via a seperate boolean)
\item Whether multiple bots are allowed in one room (boolean)
\item Whether the number of blocks in the rooms are randomized or not (boolean)
\end{itemize}

Because these XML files can be rather difficult to write by hand, there is currently a tool which creates a map for you according to certain parameters. This tool will however always generate the maps in a grid formation, with all rooms being the same size, which results in the maps always looking rather similar. \\
The map editor which we would like to construct is going to create such XML files with only drag and drop actions, so that maps become more realistic as any shape is possible, not just grid formations.