\chapter{Introduction}

In this chapter, we will explain our goals for this project.
We split them up into several tasks that work towards achieving our design
goals, which include but are not limited to: reliability, modifiability and
ease of use.

\section{Design Goals}
For this project we were given a large codebase. This codebase had grown
complex and convoluted. The task for us was to refactor the code to reduce
complexity, make it more manageable and give it a \gls{pluginarch}. \\ On
further inspection, however, it was revealed that an exact plugin architecture
wasn't precisely what the product owner wanted. By analyzing the product
owner's final goals, we decided that a modular design which can be easily
expanded or modified was actually desired. \\ 
Our goal is to have most, if not all core parts of the environment and agents
built using interfaces. This allows for easy addition and modification of
elements in the environment or agent behaviour. Creating these additional
features would then be done by creating new classes utilizing one of these set
up interfaces. Similarly these classes could then later be easily removed
without causing parts of the system to start malfunctioning or breaking
entirely. \\

For the refactoring of the code a number of tasks need to be completed:
\begin{itemize}
  \item
  \textbf{Split up codebase into Client and Server} - Currently the client and
  server share the same codebase. This makes it hard to keep a proper overview as
  it isn't always clear at a glance which class belongs to which structure, and
  some classes are even shared entirely. We want to split these structures into
  individual projects to make them easier to work with.
  \item
  \textbf{Convert to \Gls{maven} project} - There are currently no \gls{unittest} present
  in the codebase, therefore in addition to splitting the codebase we wish to
  turn these separate projects into Maven projects including Unit Tests so we can
  easily verify that the codebases are still working correctly after having made
  changes.
  \item
  \textbf{Improve and add to the \gls{gui}} - There is much room for improvement for
  the GUI. We want to make it more user-friendly in addition to adding a whole
  new GUI for the creation of maps via an easy to use drag-and-drop system.
  \item
  \textbf{Clean code} - There is a lot of unused code present, due to the
  complexity of the codebase it can be hard to tell which pieces of code are
  unused.
  \item
  \textbf{Fully utilize Repast} - Repast is a library that has been imported to
  act as the environment for the agents to act in. However due to some missing
  functionality and unfamiliarity with Repast at the initial time of implementation
  Repast isn't properly utilized, many things that have been implemented manually
  can also be handled by Repast.
  \item
  \textbf{Improve \gls{javadoc}} - Javadoc's are sometimes lacking in information or
  occasionally not present at all.
  \item
  \textbf{Remove code duplication} - There are a number of instances of code
  duplication present in the code which can make future updates harder to manage.
\end{itemize}
