\documentclass[]{article}

%opening
\title{Reflectie leerdoelen}
\author{Tim van Rossum, trvanrossum, 4246306}

\begin{document}

\maketitle
\section*{Reflectie}
Ik heb zeker geleerd hoe men in groepsverband samenwerkt aan het maken van een groot software systeem. Dit is de eerste keer dat ik samenwerk met een groep groter dan vijf man, en voor de eerste keer dat ik werk aan een software systeem dat daadwerkelijk gebruikt wordt nadat wij er mee klaar zijn. Ook noodzaakte de schaal van dit project wel tot het gebruiken van Scrum als development methode, dus ik heb ook zeker kunnen leren hoe men Scrum gebruikt om binnen korte tijd grote systemen te ontwikkelen. Verder heb ik ook verschillende principen van vakken zoals SKT en SEM in kunnen zetten tijdens het werken aan het systeem, dus dat leerdoel is ook voltooid. Het leren werken met Git heb ik voornamelijk geleerd door het aan het begin van het project zo veel mogelijk te slopen, ook dat is dus een voltooid leerdoel. Dan heb ik alleen nog het verbeteren van de communicatieve vaardigheden waar ik niet zo heel veel tijd voor heb genomen. Maar ik vermoed dat dat zeker wel goed zit: bij het MAS project vorig jaar was ik de leider van mijn groepje, en daar heb ik dan ook zeker mijn communicatieve vaardigheden kunnen verbeteren. Het verbeteren van communicatieve vaardigheden was dan ook meer bijzaak tijdens dit project.
\end{document}
