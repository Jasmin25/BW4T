\documentclass{article}

\begin{document}

\title{SIG-Lecture}
\author{Context project: Crisis management}

\maketitle
introduction blablabla 
\section*{Algemeen}
	\subsection*{Model}
	They put every method in a catogorie (4 in total) on risk category. That creates a pie chart, and that is compared to other big programs and gives a star rating of 1 till 5 
\section*{Search and Resque}
	\subsection*{Volume and Duplication}
	Use library's when possible. 
	\begin{itemize}
		\item Volume makes all issues smaller
	\end{itemize}
	\subparagraph*{Component balance and component independence}
	soort van uml diagram 
	\subparagraph*{Unit size and unit complexity}
	
	\subparagraph{verbeteringen}
	\begin{itemize}
		\item README: zorg dat hier duidelijk staat hoe het project in elkaar zit. -> component understanding 
		\item uml should  be clear, dependencies are not always needed. Discus them, make changes. 
		\item Zorg dat alle if for while dingen zoveel mogelijk gesplits wordt. 
		\item care for to many implementations in the abstract 
		\item keep your code base clean. Throw away old or decrepated stuff.
		\item keep code and library apart
		\item use meaningful names (dont always follow the checkstyle)
		\item Dont ignore exceptions(don't leave it as todo) 
		\item Describe stuf that isnt logig -> try/catch 
		\item make sure to create tests 1:1 on prodution code. This ratio should be better next time. 
		\item 
	\end{itemize}
\end{document}
