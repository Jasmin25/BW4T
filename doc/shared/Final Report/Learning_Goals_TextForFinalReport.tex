\documentclass{article}
\usepackage[utf8]{inputenc}

\begin{document}

\section{Learning goals}
\\

When we started this project, we all wrote down a couple of things we wanted to learn during this project. Catholijn told us it was more important that things were done by people who still had to learn these things, than by people who were already very good at them. So every time we had to divide some tasks, we divided them with the learning goals in mind. When someone wanted to learn how to use design patterns, we let him/her make the design patterns, and when someone wanted to learn how to test, we gave him testing tasks. \\
Of course we could not always take the learning goals in account, when we were late for a really strict deadline, or something was too hard for the person it was assigned to, a person with experience with that task would jump in.\\
But most of the time it worked out well and almost everyone reached all their learning goals. Afterwards we are really happy we did the project this way. It was certainly not the easiest way and we probably spend way more time on certain tasks than we would have spent if we would have divided the tasks according to experience. It felt illogical to give tasks to people who had barely experience in that area, while someone else could have finished that task with ease. But if we would have divided the tasks according to experience, we would never have learned as much as we learned now.    



\end{document}
