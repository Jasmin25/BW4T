\documentclass[a4paper]{article}
\usepackage{geometry}
\usepackage{amssymb}

\geometry{verbose,lmargin=2.5cm,rmargin=2.5cm,tmargin=2.5cm,bmargin=2.5cm}
\pagenumbering{gobble}
\setlength{\parindent}{0pt}

\title{Interaction Design: Usability Evaluation (group 2)}
\author{}
\date{}

\begin{document}
\maketitle

\textit{We decided to do some user tests to find out how user friendly our system really is. We based these tests on the user stories and personas described above.} \\
\\
\textbf{Preparation and expectations }\\
As we did not have much time for these user tests, we decided to
keep it short. But in order to still get representative results, we decided
to gather three test users, each representing a different persona.\\ 
Fortunately, we found 3 'ideal' test users willing to participate in
our user tests: a first year student, a professor and a researcher.\\ 
There is a lot of documentation that comes with the system so, in
real life, the users can consult these documents to learn more about
the system and how to work with it. However, these documents are too
much reading material for a simple user test. That is why we decided
not to give these documents to the test users. Also, we expected that
the test users would not need documentation or manuals because we believed
our GUI's were clear enough. \\
First we gave the participants an informed consent form (see
attachment \textit{Informed Consent}). The participants had to sign this
document before we commenced the experiment.\\
During the user tests, our participants received a short roadmap
(see attachment \textit{Manual}), with a few simple tasks they had to
complete. We used the 'Think Aloud' protocol, so we knew what our test users were thinking when performing these tasks. We wanted to know whether they immediately understood what they needed to do and how they needed to do it, or whether they needed a bit of time to comprehend such tasks. We expected the experiment to go smoothly, and that the user would not have too much trouble, but some elements that were obvious for us (because we made the system) were maybe not simple for our test users. \\
Furthermore all test users were given a questionnaire (see attachment
\textit{Questionnaire}) with seven questions in total. In addition to this question
aire, we also asked some questions in person, depending on
how the evaluation was going.\\
\\
\textbf{Test results}\\
Our test users were very positive about the changes we brought to the system. They had (some) experience with the old BW4T and they were happy with the new functionalities we had added. However, the system was not as clear and self-explanatory as we believed. \\
\\
All the test users had problems with:
\begin{itemize}
\item Finding out how to add blocks to the rooms
\item Opening a file
\end{itemize}

Opening a file went wrong because the Environment Store begins with a dialog named StartDialog. In this dialog the user can input the map's number of rows and columns, and then continue to the editor screen. Only in this screen is it possible for the user to open a file. It is only natural that this seemed contradictory to our test users: they had to set a number of rows and columns in order to open a map that already existed. \\

Moreover, a charge zone would not always appear when randomizing zones in the map. The users' opinion was that there should always be one charge zone, no matter what.\\
Furthermore, the name attributed to the randomize zones command used to be 'randomize rooms'; one of the users pointed out that this was incorrect, because not only did it randomize rooms, but it also randomized other of the map's elements. \\ \pagebreak

Finally, our users suggested various improvements for next versions of the system:
\begin{itemize}
\item A one-click-random: it should be possible to randomize everything (zones, blocks and sequence) with one button. The result should be a solvable map, so the randomize functions should keep each other's results into account. For example: there cannot be a colour in the sequence which does not exist anywhere else on the map.
\item Setting up more zones at once: it should be possible to select multiple zones by dragging the mouse over them, and then changing them all at once into rooms, blockades etc. This would be rather useful especially when dealing with large maps. 
\item Being able to save the map without a start- and/or dropzone. 
\end{itemize}

\textbf{What we did with the results }\\
Unfortunately, we did not have much time to adjust the system to the test users' suggestions. So we only focused primarily on modifications that were easy to apply. For the few things we did not have time to correct, we listed them in our final report, so that future teams that will work on this project have an idea of what there needs to be done. \\
What we changed:
\begin{itemize}
\item Added a more detailed explanation in the Environment Store. It tells the user how to edit zones and how to add blocks to rooms.
\item Made sure at least one charge zone is generated when randomizing zones. 
\item Changed the name of 'Randomize rooms' to 'Randomize zones'.
\end{itemize}

We have not yet fixed the 'open file' problem. Unfortunately, we did not have the time to do this properly, though we described this problem in our documentation for the next project group that will work on this. We also put the test users' suggestions in the final report's "Outlook" section.



\end{document}