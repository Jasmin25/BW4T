\documentclass[a4paper]{article}
\begin{document}
\subsection*{Nick}
Mijn leerdoelen waren om meer te leren over Latex, Git(Hub) en JUnit. Deze heb ik allemaal gehaald.
\begin{itemize}
    \item Ik heb in het begin van het project toen we nog wat meer aan de verslagen werkten aardig wat geleerd over Latex, bijvoorbeeld over de aard van Latex. Dus dat wanneer je Latex download, je nog niet veel kunt aangezien dtt alleen is om te compilen. Je moet dus ook een van de meerdere beschikbare editors downloaden.
Verder leerde ik hoe meerdere .tex files samen een verslag kunnen vormen, hoe je een mooie glossary aanmaakt en nog meer willekeurige dingen binnen Latex.
    \item Mijn tweede leerdoel was om te leren omgaan met Git(Hub). Dit is heel erg goed gelukt. Ik geleerd te werken met de basis operaties van Git (zoals clone, fetch, merge, add, commit and push), maar ook operaties om terug te kunnen gaan naar een eerdere versie van de code, of om sommige files te negeren bij het commiten.
Verder heb ik nog geleerd over branchen en hoe je dit het best kan doen met een groot team (en ook wat er fout kan gaan als mensen dit niet zo doen).
Ook heb ik gezien dat het met grote teams heel erg onhandig is om te veel branches te hebben, want al deze branches moeten met de master branch gemerged worden wat weer veel tijd kost, zeker als deze branches al langer bestaan.
Over GitHub zelf was niet heel veel te leren, want het was best wel voor zichzelf sprekend.
    \item JUnit was aardig makkelijk om te leren, ook omdat ik hierover nog extra moest leren door het vak Software Kwaliteit en Testen dat ik nu ook volg. Ik heb kunnen meemaken dat een JUnit test een fout in mijn code vond en dat was wel erg fijn. Helaas is dit maar een keer gebeurd, in de overige gevallen dat een JUnit test een fout gaf, was het helemaal geen fout in mijn code, maar had de code mijn nieuwe feature/verandering gewoon niet verwacht. Dit was dan wel irritant omdat je dan die test weer moest omschrijven wat weer extra werk kostte.
Hierdoor ben ik gaan vinden dat JUnit tests die kleine delen (units) van de code testen meestal alleen maar extra werk kosten, en bijna niks opleveren. Wat ik wel nuttig vind zijn integratie tests. Dus het systeem een input meegeven en kijken of de output hetzelfde blijft na veranderingen van de code. Dat het systeem van binnen veranderd en dus wat unit tests kapot maakt is niet echt belangrijk, zolang de uiteindelijke gewenste output maar goed gegenereert wordt.
\end{itemize}
\end{document}