\section{Environment store}
In the beginning of the project the customer made clear what their desires were for the Environment Store. These are listed below:
\begin{itemize}
	\item Random Map Generator
	\begin{itemize}
		\item Number of Maps
		\item Number of Rooms
		\item Number of Bots
		\item Blockades
		\item Without constraints like of for example a maximum of 26 columns
	\end{itemize}
	\item Make a map by hand
	\item Save environments
\end{itemize} 

All these things are roughly implemented as they are described. 
We will walk them through, not completely in order. \\
\indent The creating of maps by hand is changed completely. Before we changed it it simply asked for a number of rows and columns and then generated rows of rooms with a drop zone on the bottom. After that you could determine what blocks every room contained and in what sequence they had to be collected. All that is left of this is the grid which is generated by the number of rows and columns that are specified. This grid is empty, this means that every zone is a corridor. Then rooms can be added in every zone by hand. Also the spawn points of the bots can be determined and the drop zone can also be placed wherever the user likes. Extra additions are blockades and charging zones. The charging zones are added because the bots can now have batteries which need to be charged somehow. The reason we implemented it like this is because the grid could easily be adapted to a fully custom made map. The maximum number of rows and columns is not removed but it is raised to 100. We did not raise it to unlimited because it would simply get too big and you could not get a clear overview of the map that way.\\
\indent The 'random map generator' is an option in the map editor. At this point the number of rows and columns of the grid is already specified. In this generator the rooms are still added like rows of rooms and not completely random locations because you would get too many unsolvable maps. Every time you generate the map you get a slightly different amount of rooms. This was done to improve the randomness of the generated maps. The number of bots is also not generated here. We chose not to do that because the bots can be added in the Scenario Editor. We left out the option to automatically create a certain amount of maps at once. This is because the way we do it now you can generate a map and after that you can easily make additional changes. Of course the maps the user creates can be saved and it can be easily loaded in the BW4T environment.