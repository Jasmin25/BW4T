\textbf{Learning Goals Reflection Jan Giesenberg}

During the project I learned different methods of how to work together with multiple people on the same files. When programming it is very easy to integrate the work from different people using a versioning system, in our case we used git in conjunction with the website github which acts as central point to upload the most recent changes to. When working with documents containing tables or text there are multiple options. If the document is a report it is easiest written in \LaTeX which can be compiled into many different file formats. When working with excel type files using google docs is the easiest, as it allows multiple people to change the document in real-time.

I made a lot of progress in understanding which problems to fix when faced with refactoring a large code base. For instance it is very important to do your changes in the correct order. To rewrite parts of a class without reorganizing the general structure first does not make sense. First test to check the working state of the programm need to be written, then the organization of the project should be improved then the individual problems need to be asserted and last of all you should work on documenting the changes.

Working in a SCRUM team has been a very challenging experience, there are many subtle problems that can creep up to you. For instance when the scrum master does not show enough leadership the team will stagnate and not get anything done, as tasks are just postponed to the next and the next week. A lot of time gets lost with what should be short tasks.

I was able to get the experience of being a scrum master for a week as our scrum master went on vacation. I was quite happy doing the extra management work and I felt that we were making more progress than in the weeks before as I made sure that team member knew they just needed to ask if they were stuck on a problem instead of trying to work it out for weeks on end.

I did get a little further on trying to get people to find solutions to problems they ask me about. The strategy I implored was to ask what they had thought up so far and then tried to lead their thinking to a solution by asking more in-depth questions. Basically not directly telling them the answer but leading their thoughts through a correct path of thinking. This method proved quite successful for most questions I got, but some of them I did not have an answer myself so I included them in my search, so they could see and adopt the strategies for searching something.
