\section{Old maps vs new maps}
The old map editor has been completely redesigned to what we now call the Environment Store. 
It offers the user the ability to create much bigger maps and place the different types of zones 
across the map according to a grouping that is preferred for a simulation by the user.

\subsection{Differences old and new maps}
The main difference in the maps that are generated by the new Environment Store is that the distribution
of rooms and other zones on the map is no longer set through a standard algorithm but can be set by the user themselves.
Through a grid of equally sized zones you can set each of these zones to being either a corridor, room, start zone, drop zone, blockade or charging zone.
This way we allow the user to have a lot more variety in the maps they create and allow for the possibility to
create a map that can more accurately simulate a real environment.
\\
\\
The old maps always used to have the start- and drop zone at the bottom, this is no longer a must in the new maps.
Start- and drop zones, as well as any other type of room can be placed anywhere in the map. This is because the
robot's starting position is not necessarily the place where the blocks are supposed to be dropped.
\\
\\
Also, you can now add two new types of zones called blockades and charging zones. 
The blockades serve to create a map with parts where the robot should not be able to pass.
This way the user can create a maze or zones that should be difficult or take a long time to reach. 
We have added charging zones which allow the bots that contain a battery to recharge when passing it. 
A charging zone is like an open space and does not contain any walls or doors. Both blockades and 
charging zones are the size of a basic room or corridor and therefore easily fit in the new mapping.


\subsection{Editing saved maps}
In the old map editor the user was never able to modify a previously created map, while in the new Environment Store they can.
This allows the user to create a map, save different versions of it and edit it in a later stage when an error was spotted after the map had been saved.
\\
\\
The user can also open maps that were created using the old map editor. When the user does this, the map is automatically converted to be ready to edit in the Environment Store. This does mean that the large drop zone, which in these maps is located at the bottom of the map, is set
to the standard size of the grid. Also it creates corridors in the zones left and right from the old drop zone which used to be non-navigable empty zones.
Besides these minor changes the map will be the exact same as after saving it in the old map editor.

\subsection{Benefits}
The benefits of the new maps, that can be created in the Environment Store, have been listed below:

\begin{itemize}
	\item The user is no longer bounded to a standard mapping of rooms.
	\item The user can now create a map that is up to 4 times the size of the largest possible map created in the old editor.
	\item The user can add blockades to the map.
	\item The user can add charging zones to the map.
	\item The user can save and open previously created maps.
	\item The user can randomize the positions of rooms, the blocks in a room and the target sequence separately using extra options.
	\item The number of entities that can spawn in a map is no longer bound in the new maps.
\end{itemize}


\subsection{Disadvantages}
Although we have tried to improve all aspects, there are still minor disadvantages to the new Environment Store. Those are listed below:

\begin{itemize}
	\item It generally takes more time to create a map in the new Environment Store.
	\item To set the number of bots that can spawn on a map to infinite, the bots will spawn on top of each other which could go unnoticed by the user.
\end{itemize}

\subsection{Considerations}
One thing we have considered is how we were going to spawn the bots in the new maps. We came to the conclusion that when physical realism has been turned
on in the Scenario Editor, we would still like to be able to spawn as many bots as the user wants in the map. This is why we have chosen to create 4 spawn points per
start zone, allowing us to spawn up to 4 bots of the largest size without letting them collide. A fifth bot would simply spawn on the same place as the first one after that one has moved. This should
not be possible when physical realism has been turned on, therefore we have set the functionality that as long as they spawn on top of each other they can move, but as soon as
they have moved away from each other they should not be able to move over each other again. By doing this, the number of bots that can spawn in a map is not bounded by the number
of start zones.